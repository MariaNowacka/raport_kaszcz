\documentclass[12pt]{article}
\usepackage[utf8]{inputenc}
\usepackage[T1]{fontenc}
\usepackage{graphicx}
\usepackage{amsmath}
\usepackage{hyperref}
\usepackage{geometry}

\geometry{
    top=1.8cm,
    bottom=2.5cm,
    left=2.5cm,
    right=2.5cm
}
\title{Kompuuterowa Analiza Szeregów Czasowych \\ \textbf{Analiza danych rzeczywistych przy pomocy modelu ARMA} }
\author{Prowadzący: mgr inż. Justyna Witulska \\  Autorzy: Zuzanna Nasiłowska i Maria Nowacka}
\date{31 stycznia 2025 r.}

\begin{document}
\maketitle

\section{Wstęp}
\subsection{Cel pracy}
\subsection{Dane}
Dane użyte w tej analizie pochodzą z:
\\link\\ Zostały zebrane w okresie XXXX i opisują pomiary XXXXX \\
\subsection{Wizualizacja danych}

\section{Przygotowanie danych do analizy}
\subsection{Badanie jakości danych}
% (opcjonalnie) wyodrębnienie z danych obserwacji do zbioru testowego,
\subsection{Dekompozycja szeregu czasowego}
– wykres ACF oraz PACF dla surowych danych, \\
% – (opcjonalnie) test ADF weryfikujący hipotezę o niestacjonarności dla surowych danych (Augmented Dickey-Fuller Test), \\
– identyfikacja trendów deterministycznych: metody omawiane na wykładzie oraz na laboratorium -
(zadanie 4, lista 5) / różnicowanie / różnicowanie sezonowe / transformacje stabilizujące wariancję
(Boxa-Coxa), \\
– wykres ACF oraz PACF dla uzyskanego szeregu, \\
% – (opcjonalnie) test ADF weryfikujący hipotezę o niestacjonarności dla uzyskanego szeregu (Augmented Dickey-Fuller Test). 

\section{Modelowanie danych przy pomocy ARMA}
\subsection{Dobranie rzędu modelu}
\subsection{Estymacja parametrów modelu}

\section{Ocena dopasowania modelu}
\subsection{Przedziały ufności}
\subsection{ porównanie linii kwantylowych z trajektorią}
% (opcjonalnie) prognoza dla przyszłych obserwacji i porównanie z rzeczywistymi danymi.


\section{Weryfikacja założeń dotyczących szumu}
\subsection{Średnia}
\subsection{Wariancja}
\subsection{Niezależność}
\subsection{Normalność rozkładu}


\section{Zakończenie}
\subsection{Podsumowanie}
\subsection{Wnioski}


\end{document}